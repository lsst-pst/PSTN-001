\documentclass{emulateapj}


\slugcomment{DRAFT Version 0.1 of Aug 16, 2017}


\newcommand\x         {\hbox{$\times$}}
\newcommand\othername {\hbox{$\dots$}}
\def\eq#1{\begin{equation} #1 \end{equation}}
\def\eqarray#1{\begin{eqnarray} #1 \end{eqnarray}}
\def\eqarraylet#1{\begin{mathletters}\begin{eqnarray} #1 
                  \end{eqnarray}\end{mathletters}}
\def\mic              {\hbox{$\mu{\rm m}$}}
\def\about            {\hbox{$\sim$}}
\def\Mo               {\hbox{$M_{\odot}$}}
\def\Lo               {\hbox{$L_{\odot}$}}
\def\comm#1           {{\tt (COMMENT: #1)}}
\def\kms   {\hbox{km s$^{-1}$}}

\usepackage{graphicx}
\usepackage{xspace}

\usepackage[usenames]{color} 
%\newcommand{\B}[1]{{\color{blue} #1}}
%\newcommand{\R}[1]{{\color{red} #1}}
\newcommand{\G}[1]{{\color{red} #1}}
\newcommand{\B}[1]{{#1}}
\newcommand{\R}[1]{{\color{red}}}

\begin{document}
\title{On the Choice of LSST Flux Units (*** draft ***)} 
\author{\v{Z}eljko Ivezi\'{c} and the LSST Project Science Team} 


\begin{abstract}

A linear measure of flux is preferred for LSST catalogs. This document
provides some technical detail about this issue and proposes to adopt
nano-Jansky as the standard LSST flux unit.  

\end{abstract}


\section{Introduction} 

A linear measure of flux, not logarithmic magnitude scale, is preferred
for LSST catalogs (e.g. forced fluxes can be negative due to stochastic
background fluctuations). The choice of a particular flux unit is often
conflated with issues of the interpretation of broad-band photometry 
and systematic uncertainties in photometric calibration. This document
summarizes most important technical details about these issues and 
proposes to adopt nano-Jansky as the standard LSST flux unit. 

Relevant technical discussion is provided in \S 2. Readers familiar 
with broad-band photometry and photometric calibration can skip
directly to \S 3, where  arguments are laid out for adopting nanoJansky 
as the LSST flux unit. 


\section{What flux will LSST measure?} 


\subsection{CCDs count photons} 

CCDs don't measure energy flux - CCDs count photons over some wavelength range set 
by the overall atmosphere plus system throughput,  $S_b(\lambda)$ (defined as the probability 
that a photon with wavelength $\lambda$, or frequency $\nu=c/\lambda$,  will be 
transmitted through Earth's atmosphere and through the observing apparatus, and converted
into an electron, including the effects of both optics, sensors and all other potential losses). 
This quantity is {\it not}  defined per unit energy, or per unit wavelength (or frequency): 
it is simply a dimensionless probability -- a number between 0 and 1. 

Given a flux of photons per unit time, area and frequency interval, $N_\nu$, the 
source counts, $C_b$ (in ADU), are proportional to 
\begin{equation}
\label{Nnu}
        C_b \propto \int N_\nu(\nu) \, S_b(\nu) \, d\nu,
\end{equation}
with the constant of proportionality increasing with effective collecting area and exposure 
time (index $b$ stands for ``bandpass''). 

The integration in eq.~\ref{Nnu} is over frequency $\nu$ because the photon flux $N_\nu$ 
is expressed per unit frequency. However, the running variable can be either  $\nu$ or 
$\lambda$, and the choice of $\lambda$ is more convenient in this context. It is 
easy to show, using $|d\nu/\nu| = |d\lambda / \lambda|$ (which follows from $\nu \lambda=c$),
that an equivalent form of  eq.~\ref{Nnu} is
\begin{equation}
\label{Nnu2}
         C_b\propto \int N_\nu(\lambda) \, S_b(\lambda) \, \lambda^{-2} \, d\lambda. 
\end{equation}


CCDs count photons but (unfortunately) don't record the photons' wavelength/frequency/energy.
Nevertheless, it is possible to relate (calibrate) the measured source counts to energy flux -- though
with a few important caveats, as described below. 

 

\subsection{Definition of the specific flux} 

Let us first  define $F_\nu$: the specific flux (flux per unit frequency, $\nu$) of an 
object {\it at the top} of Earth's atmosphere. The SI units for $F_\nu$ are 
W m$^{-2}$ Hz$^{-1}$ (= 10$^{3}$ erg cm$^{-2}$ s$^{-1}$). Because astronomical 
fluxes are small, the IAU defined in 1973 a more convenient unit\footnote{The
name honors radio astronomer Karl Jansky. Papers about the 3$^{rd}$ Cambridge 
catalog of quasars published in 1950's already used $10^{-26} \, {\rm W m^{-2} Hz^{-1}}$
as a standard flux unit (without calling it Jansky).},  Jansky (Jy): 
\begin{equation}
            1 \, {\rm Jy} = 10^{-26} \,\, {\rm W m^{-2} Hz^{-1}}.
\end{equation} 

The specific flux can also be defined per unit wavelength, $F_\lambda$, using energy 
conservation $F_\nu |d\nu|=F_\lambda |d\lambda|$ and $\lambda \nu = c$.
The choice of $F_\nu$, as opposed to $F_\lambda$, makes the flux conversion 
to the AB magnitude scale (see below) more transparent, but otherwise is completely
arbitrary. Similarly, the running variable can be either  $\lambda$ or $\nu$, and the 
choice of $\lambda$ is more convenient.



\subsection{From counts to the specific flux} 

The flux of photons, $N_\nu$ (s$^{-1}$ m$^{-2}$ Hz$^{-1}$), is related to the specific flux $F_\nu$ as
\begin{equation}
\label{FNnu}
                 F_\nu = h \nu \times N_\nu,  
\end{equation}
where $h\nu$ is the energy of a single photon with frequency $\nu$. Therefore,
eq.~\ref{Nnu2} can be rewritten as
\begin{equation}
\label{Nnu3}
         C_b\propto \int F_\nu(\lambda) \, S_b(\lambda) \, \lambda^{-1} \, d\lambda. 
\end{equation}


\subsection{Definition of broad-band flux} 

As discussed in the LSST Science Requirements Document (Section 3.3.4), ``LSST is a 
broad-band photometric system and will deliver calibrated in-band flux, $F_b$, 
defined by 
\begin{equation}
\label{Fb}
              F_b = \int{F_\nu(\lambda) \phi_b(\lambda) d\lambda},
\end{equation}
where $F_\nu(\lambda)$ is the specific flux of an object {\it at the top} of the atmosphere, 
and $\phi_b(\lambda)$ is the normalized system response for the given band.'' This
expression follows directly from eq.~\ref{Nnu3}. 

The normalized system response is defined by 
\begin{equation}
\label{PhiDef}
\phi_b(\lambda) = {S_b(\lambda) \lambda^{-1} \over \int{S_b(\lambda) \lambda^{-1} d\lambda}}.
\end{equation}
where $S_b(\lambda)$ is the overall atmosphere + system throughput.  It is defined by
\begin{equation}
\label{SDef}
         S_b(\lambda) = S_b^{sys}(\lambda) \times S_b^{atm}(\lambda),
\end{equation}
where $S_b^{atm}(\lambda)$ is the {\it probability} that a photon 
with wavelength $\lambda$ will be transmitted through the atmosphere, and
$S_b^{sys}(\lambda)$ is the probability that the transmitted photon will be 
converted into an electron by the system. Again, these quantities are
dimensionless probabilities (numbers between 0 and 1). 

As discussed above, the $\lambda^{-1}$ factor in eq.~\ref{PhiDef} reflects the fact that CCDs are 
photon-counting devices: it comes from the conversion of energy per unit frequency
into the number of photons per unit wavelength. 

Note that the product $\phi_b(\lambda) d\lambda$ is dimensionless; it acts in eq.~\ref{Fb} as a 
dimensionless weighting function and the unit for $F_b$ is {\it same} as for $F_\nu$
(by construction).  In other words, eq.~\ref{Fb} doesn't represent ``integration under the 
$F_\nu$ curve'' -- instead, $F_b$ is a ``weighted average'' of $F_\nu$ ($N_\nu$ is 
``integrated under the curve'', not $F_\nu$). 

Had we chosen to use $F_\lambda$ instead of $F_\nu$, an analog
to eq.~\ref{Fb} would have been 
\begin{equation}
\label{FbL}
              F_b^\ast = \int{F_\lambda(\lambda) \phi^\ast_b(\lambda) d\lambda},
\end{equation}
where
\begin{equation}
\label{PhiL}
\phi^\ast_b(\lambda) = {S_b(\lambda) \lambda \over \int{S_b(\lambda) \lambda d\lambda}}.
\end{equation}
In this case, the calibrated flux $F_b^\ast$ has the same units as $F_\lambda$, and
the weighting factor $\phi^\ast_b(\lambda) d\lambda$ is now skewed more towards 
the red edge of the bandpass, compared to $\phi_b(\lambda) d\lambda$. Of course, 
$C_b$ is same in both cases - it is only our (arbitrary) choice of flux calibration that 
distinguishes $F_b$  and $F_b^\ast$. The only practical implication of the choice between 
$F_b$  and $F_b^\ast$ is the type of the spectral energy distribution for which photometric 
standardization correction for bandpass variation (see eq.~\ref{FbStd}) vanishes.  


\subsection{The curse of broad-band flux} 

There are a few consequences of the finite width of $\phi_b(\lambda)$ that need to be 
emphasized (and are {\it not} the result of the specific choice for flux calibration). 

Even for a temporally non-variable $F_\nu(\lambda)$,  $F_b$ will vary if 
$\phi_b(\lambda)$ varies (even if the atmospheric and system properties
are unchanged, variation of observing airmass can change $\phi_b(\lambda)$). 
In order to standardize measurements to a 
common standard system, one in which $F_b$ of a temporally non-variable 
source would not vary (modulo random noise),  we need to define a {\it standard}
normalized system response, $\phi^{std}_b(\lambda)$. In addition, we must know the {\it shape} 
of the source spectral energy distribution (SED), $f_\nu(\lambda)$, defined by 
\begin{equation}
        F_\nu(\lambda) = F_o \, f_\nu(\lambda),
\end{equation} 
where $F_\nu(\lambda_o) = F_o$ and $f_\nu(\lambda_o) = 1$ for some fiducial 
wavelength $\lambda_o$. Then we can compute standardized flux as 
\begin{equation}
\label{FbStd}
F^{std}_b =  F_b \, {\int{f_\nu(\lambda)  \phi^{std}_b(\lambda) d\lambda }  \over   \int{ f_\nu(\lambda) \phi_b(\lambda) d\lambda }}. 
\end{equation}
Traditionally, corrections to the standard system are called {\it color terms}. Historically,
they were obtained empirically (typically as linear functions of source color and airmass)
rather than by using eq.~\ref{FbStd}. Assuming main-sequence stars and standard atmosphere, 
plausible variations of airmass induce variations of $F_b$ around $F^{std}_b$ of a few percent. 

Without knowing, or assuming, $f_\nu(\lambda)$, it is {\it mathematically impossible}  to 
standardize measurements $F_b$ -- this is the ``curse'' of broad-band fluxes
(note that in the special case of a flat SED, $f_\nu(\lambda)$ = constant, $F^{std}_b =  F_b$;
had we chosen $F_b^\ast$ instead of $F_b$, the standardization correction would vanish for
$f_\nu(\lambda) = \lambda^2$, that is, for $f_\lambda(\lambda)$ = constant). 
This ``curse'' cannot be avoided whatever is the flux calibration choice (i.e. $F_\nu$ vs. $F_\lambda$);
the ``problem'' is that CCDs don't measure integrated flux -- they count photons and
don't record the photons' wavelength.


\subsection{Standardized fluxes} 
 
For each flux measurement, LSST will report both $F_b$ and $\phi_b$ (with $\phi^{std}_b(\lambda)$ 
pre-defined and always known). There is also need to report standardized flux
computed using eq.~\ref{FbStd} (e.g. to help users construct color-color and color-magnitude
diagrams, or to search for variable sources). There are at least two options for choosing $f_\nu(\lambda)$:
i) assume a flat SED ($F^{std}_b =  F_b$, i.e. effectively no correction is applied), and ii) assume 
the best possible estimate of object's SED, using available LSST color measurements and possibly 
other information. 

Neither choice is perfect: the first choice, while simple, does not account for the variation of $F_b$ around
$F^{std}_b$ due to changes of $\phi_b$ (except in case of flat SED), while the second choice
can be grossly incorrect (e.g. when SED type is incorrectly chosen, such as stellar SED instead
of quasar or supernova SED). Therefore, it is important to enable users i) to undo whatever default 
flux standardization correction was used, and ii) to easily re-do the computation with a different choice 
of the spectral energy distribution (e.g. for multiple hypothesis testing, such as distinguishing
``star'', ``quasar'', and ``supernova'' SEDs, or galaxy SEDs of different intrinsic types and at 
different redshift). The current baseline LSST plan is option ii). 

% LSST will provide corresponding bandpass for each photometric measurement, while the
% normalization will be fixed by providing flux under the assumption of flat SED. 
% DO WE NEED TO SEPARATE THESE TWO TERMS? 


\subsection{Some pitfalls when interpreting measured fluxes}

When comparing a model for the specific flux, $F_\nu^\ast(\lambda)$, to  
measurements $F_b$, the proper way to proceed is to compute the model prediction 
for $F_b$ using eq.~\ref{Fb}
\begin{equation}
\label{FbModel}
             F^{model}_b = \int{F_\nu^\ast(\lambda) \phi_b(\lambda) d\lambda},
\end{equation}
and then compare $F^{model}_b$ to measurement $F_b$.  
When measurements $F_b$ have already been standardized as $F^{std}_b$, 
this data vs. model comparison (e.g. photometric redshift estimation) can 
suffer from systematic errors when SED shape, $f_\nu(\lambda)$,  assumed 
for flux standardization differs from the shape of $F_\nu^{model}(\lambda)$,
{\it even when} $\phi_b(\lambda)$ was substituted by $\phi^{std}_b(\lambda)$. 

Flux measurements ($F_b$ or $F^{std}_b$) are often interpreted as corresponding to 
$F_\nu(\lambda_{eff})$, at some effective wavelength, $\lambda_{eff}$, and compared 
to model flux $F_\nu^\ast(\lambda_{eff})$. This practice usually results in systematic
errors because $\lambda_{eff}$ is a function of $f_\nu(\lambda)$, and thus there
is no universal value of $\lambda_{eff}$ applicable to all sources. 


\subsection{Calibration of counts to get fluxes} 

Image processing pipelines, more precisely object measurement pipelines/algorithms,
will produce counts, $C_b$ (together with $\phi_b$). It is assumed that all the relevant 
instrumental and other effects had been taken into account such that the following 
relationship is valid
\begin{equation} 
               F_b = \alpha \,  C_b,
\end{equation} 
for all sources from some judiciously chosen ``calibration patch'' (spatial variation of $\alpha$ over
the patch is assumed to be corrected for, though the above equation could be easily
generalized; in principle, this  ``calibration patch'' could correspond to the entire 
sky if all $C_b$ were ``reduced to the same system'' using self-calibration). 

A set of calibration stars will be available to estimate $\alpha$: for these stars 
we (assume that we) will know broad-band flux in an arbitrary calibration bandpass
$\phi_{calib}(\lambda)$ (e.g. in Gaia's bandpasses)
\begin{equation}
\label{FbCalib}
             F_{calib} = \int{F_\nu(\lambda) \phi_{calib}(\lambda) d\lambda}.
\end{equation}
We also assume that we will have a good knowledge of the SED shape,
$f_\nu(\lambda)$ for calibration stars. The implied fluxes of calibration stars corresponding 
to bandpass $b$, $F_b^{calib}$ can then be computed analogously to eq.~\ref{FbStd}, using 
$\phi_b(\lambda)$ and $\phi_{calib}(\lambda)$,
\begin{equation}
\label{FbStdCalib}
F^{calib}_b =  F_{calib} \, {\int{f_\nu(\lambda)  \phi_b(\lambda) d\lambda }  \over   \int{ f_\nu(\lambda) \phi_{calib}(\lambda) d\lambda }}. 
\end{equation}


Finally, the calibration coefficient $\alpha$ (corresponding to photometric zeropoint when
working in magnitude space) is computed from
\begin{equation} 
                 F_b^{calib} = \alpha \,  C_b^{calib},
\end{equation} 
by the usual least squares minimization, or perhaps using a more robust statistical method. 


\subsection{AB magnitudes} 
Traditionally, the in-band flux is reported on a magnitude scale, and LSST has adopted
AB magnitudes defined as 
\begin{equation}
\label{ABmag}
              m^{AB}_b = -2.5\log_{10}\left({F_b \over F_{AB}}\right).
\end{equation}
where $F_{AB}$ = 3631 Jy. 
The same expression applies to $F^{std}_b$, or any other flux. 



\section{The choice of flux unit}  

Most astronomical surveys, especially space-based surveys and surveys
at wavelengths other than optical, used Jansky as the flux unit.  One notable
exception is SDSS, which used maggies. 


\subsection{The curious case of maggies} 

In analogy with eq.~\ref{ABmag}, magnitudes can be defined\footnote{
See https://www.sdss3.org/dr8/algorithms/magnitudes.php} as 
\begin{equation}
\label{maggie}
               m \equiv - 2.5\log_{10}\left({\rm maggie}\right),
\end{equation}
where ``maggie'' is the source flux expressed in some arbitrary
units,
\begin{equation}
\label{maggie2}
               {\rm maggie} \equiv {{\rm flux } \over F_o}. 
\end{equation}
In case of AB magnitudes, eq.~\ref{ABmag} implies that maggie is 
flux measured in units of 3631 Jy, 
\begin{equation}
\label{eq:defmaggie} 
           {\rm maggie} = {{\rm flux \, (Jy)} \over {\rm 3631 \, Jy}}. 
\end{equation} 
In practice (e.g. SDSS), a more convenient
quantity is nano-maggie, which is flux measured in units of 3631 nanoJy.


Maggies were introduced as an alternative flux unit in order to emphasize
that zeropoints for astronomical flux calibration change often -- and 
presumably much more often than the actual counts measurement. Such 
was the case of SDSS, where the sky was imaged essentially once and with
a precision better then the accuracy of calibration photometry (i.e. at 
the bright end systematic photometric uncertainties were larger than 
random uncertainties). It was anticipated that photometric zeropoints 
would eventually improve and the dataset recalibrated.  But it needs to 
be noted that fluxes also change when other aspects of calibration change
(non-linearity and cross-talk corrections, flatfields, standard apertures, 
point-spread function, etc.) -- photometric zeropoints are only one of 
many calibration factors. Indeed, it turned out that in practice most 
SDSS users would simply convert maggies to Jansky and AB magnitudes
(and the SDSS Project curated a list of five best estimates of the systematic
flux offsets in the $ugriz$ bands introduced by this conversion). 
 
It is sometimes claimed that maggies have the benefit of not pretending 
to exactly correspond to physical units (Jy). This somewhat philosophical
advantage was not born in practice because of the immediate users' 
flux conversion to Jansky mentioned above. 


\subsection{Case for Jansky as the preferred flux unit} 

There is fundamentally no difference between the flux measurements discussed
here and any other physical measurement that is subject to random and systematic
uncertainties. It is rather common to report both types of measurement uncertainties
in physical sciences. Indeed, the LSST  Science Requirements Document (Section 3.3.4)
explicitly addresses the issue of systematic uncertainties in photometry and
introduces the separation of ``internal absolute'' calibration accuracy 
from ``external absolute'' ” calibration accuracy and from the offset from the
overall flux scale. The deviation of the LSST system from a perfect AB system
(that is, the systematic error in calibrating the flux scale to correspond to Jansky)  
$\Delta_m$, is expressed relative to the fiducial $r$ band as 
\begin{equation}
             \Delta_m = \Delta_r + \Delta_{mr}, 
\end{equation}
(eq.~9 from the SRD), where $\Delta_r$ describes overall systematic uncertainty 
in the LSST flux scale calibration (a single number for the whole survey). The 
``band-to-band zeropoint errors'' $\Delta_{mr}$ are thus decoupled from the overall 
``gray scale'' offset $\Delta_{r}$, which minimizes error covariances. Similarly to SDSS, 
a variety of methods will be used to assess likely values of $\Delta_{mr}$ and 
$\Delta_r$ for each LSST Data Release. 

To summarize, the case for Jansky as the preferred LSST flux unit, as opposed
to maggies, derives from the following arguments: 

\begin{enumerate}
\item There isn't that much difference between the two options; in particular,  the key 
decision is to report physical flux on a linear scale and whether the flux unit is Jy 
or 3631 Jy is of secondary importance.  Since Jy is widely used in astronomy, it seems 
wise to give it preference over much less utilized maggies. For the faint fluxes
probed by LSST, a more convenient unit is nanoJansky (nJy). The AB magnitude 
values of 27.5 (fiducial coadded image depth)  and 24.5 (fiducial single-image depth) 
correspond to 36.3 nJy and 575 nJy. 

\item Irrespective of which flux unit is chosen, the changes of the photometric 
calibration zeropoints will affect whatever is reported, whether it is maggie, $mag^{AB}$, 
$mag^{Vega}$, or nJy, as long as it is implied that physical flux is reported. Only if
truly relative measures of flux are reported (e.g. with respect to Vega, but without knowing 
what calibrated Vega fluxes really are), will magnitudes (but not fluxes in maggies or 
nJy) stay unchanged. However, in this case rather ugly details are hidden (e.g., how 
exactly was the relative flux calibrated and how stable is the flux scale), and in the context 
of sub-1\% photometry this case appears to be of historical interest only. 
\item ``But: we are measuring broad-band flux, and not the specific flux!''  Well, 
yes, this is an unfortunate fact. However, it is not relevant for the maggie vs. Jy discussion. 
It is impossible (mathematically!) to relate these two flux measures without knowing 
the source SED, whether we use magnitudes, maggies or Jy! 
\item When an estimate of the source SED is available, the transformation from
the broad-band flux to the specific flux is dimensionless and thus obviously independent 
of the choice for flux units (again, the broad band flux vs. the specific flux issues do not provide
arguments for maggie vs. Jy discussion).
\end{enumerate}

In conclusion, the proposal is to calibrate LSST fluxes to a physical scale (e.g. using
Gaia catalogs), adopt nJy as the flux unit, and report the best available estimates of 
$\Delta_{mr}$ and $\Delta_r$ with each LSST Data Release. In particular, $\Delta_r$ 
will describe the systematic flux uncertainty --  the discrepancy between an ideal 
flux scale in Jansky and ``Jansky'' scale reported by LSST. 



{\it Acknowledgments:} this document has greatly benefited from discussions between
the LSST Project Science Team members and Tim Axelrod, Michael Blanton, David 
Burke, Arjun Dey, Mark Dickinson, Doug Finkbeiner, Lynne Jones,  Jeff Newman, 
John Parejko, David Schlegel, and Peter Yoachim,



\end{document} 



Arjun: 
The passion on this subject is quite remarkable - and the boundaries drawn 
appear to be whether someone was brought up inside SDSS or not. 
The argument here is about the zero point of the quoted number, not the principle. 

Jeff Newman: the Jansky was defined to be 10^-26 W/m^2/Hz by the IAU in 1973.
Note: the 3C people used 10^-26 W/m^2/Hz as the standard unit in their papers 
in the 1950’s, without calling them Janskys. 

Fink: If we are looking for the simplest possible linear flux system that is related 
to what astronomers know and love, and retains a hint to the user that broadband 
photometry is broadband AND is poorly tied to physical units, mgy make a lot of 
sense.  To use Jy is to overload the definition of Jy rather badly. 

ZI: there are always random and systematic errors. 

Arjun: I agree with Jeff and Zeljko … we can either spend time saying "well, but they aren’t 
*really* Jy, they are just sorta Jansky-like things” or saying “well, these are nanomaggies and 
to convert to something like AB magnitudes, use a constant of 22.5, but these are not really 
AB magnitudes but AB magnitude like things”.


ZI thinks these are fallacies: 

Maggies also have the benefit that they don’t pretend to exactly correspond
to X watt/s.  If one is in Jy, one would be obligated to change all your fluxes
if you get a better understanding of your zero-points.

same for maggies: they are defined in flux scale too

> concern about fluxes changing when you better understand your zero points.  
> that’s handled with magnitudes; if you want something linear, use maggies 

but  magnitudes also change when other aspects of calibration change (linearities, 
standard apertures, etc.) - zero points are the only factor here 


if maggies are used, most people will just convert them to microJy or AB anyway. 

if mags are -2.5log(f/f0), maggies are just f/f0,
   
      magnitude = -2.5*log(maggie) 

nanomaggies doc: 
https://www.sdss3.org/dr8/algorithms/magnitudes.php
http://data.sdss3.org/datamodel/files/PHOTO_SWEEP/RERUN/calibObj.html

If you want to see the sorts of misconceptions one can be led into by
thinking about broad-band photometry wrongly, see this:
http://wise2.ipac.caltech.edu/docs/release/prelim/expsup/sec4_3g.html#FluxCC


Fink: somehow maggie accounts for the broadband flux, but Jy doesn’t?? 
Our backends do *not* give us \int_{band} I_\nu d\nu.  If they did, I would push
for Jy.  Instead, the quantity they give us relates to \int_{band} I_\nu d\nu in a way 
that depends on the SED of the source.  

Another misunderstanding: 
> I must say that this is a good argument for nMgy. I have used such units in the past, 
> and the convenience of the one line transformation into magnitudes with a constant 
> 22.5 zero point is pleasant.
















In order to interpret photometric measurements at the error level
specified below ($<$1\%), both $S_b^{sys}(\lambda)$ and $S_b^{atm}(\lambda)$
must be known with sufficient precision. Experience with precursor surveys, 
such as SDSS, suggest that the dependence of both functions on
wavelength have to be directly measured to break the 1\% photometric error 
barrier (especially for sources with complex spectral energy distributions,
such as supernovae). While the individual normalizations of $S_b^{sys}(\lambda)$ and 
$S_b^{atm}(\lambda)$ are {\it not} required (c.f. eq.~\ref{PhiDef}) to 
interpret measurements using eq.~\ref{Fb}, the flux scale (calibration) 
errors affect the reported values of $F_b$ (i.e. $m_b$). Therefore, 
for each photometric measurement both $F_b$ and $\phi_b(\lambda)$ 
will be reported, together with their estimated uncertainties. These 
two quantities will capture fundamental information included in LSST
measurements, and will enable accurate transformation of $F_b$ 
to systems with similar $\phi_b(\lambda)$ when the source spectral
energy distribution is known or assumed. In particular, corrections
to some standardized ``average'' LSST system, $\phi_b^{std}(\lambda)$,
will be defined during the commissioning period for the most relevant
spectral energy distributions (such as main sequence stars and normal 
galaxies). 


The requirements for photometric calibration accuracy are specified using 
the following error decomposition (valid in the limit of small errors)
\begin{equation}
\label{photoSysErr}
 m^{std}_{cat} = m^{std}_{true} + \sigma + \delta_m(x,y,\phi^{std},\alpha,\delta,SED,t) + \Delta_m,
\label{eq:mstdcat}
\end{equation}
where $m^{std}_{true}$ is the true magnitude defined by eqs.~\ref{Fb} and
\ref{ABmag}, $m^{std}_{cat}$ is the cataloged LSST magnitude, \R{both evaluated using
$\phi^{std}$,} $\sigma$ is the random 
photometric error (including random calibration errors and count extraction
errors), and $\Delta_m$ is the overall (constant) offset of the 
internal survey system from a perfect AB system (the six values of $\Delta_m$ 
are equal for {\it all} the cataloged objects). Here, $\delta_m$ describes 
the various systematic dependencies of the internal zeropoint error around 
$\Delta_m$, such as position in the field of view ($x,y$), the normalized 
system response ($\phi$), position on the sky ($\alpha, \delta$), and
the source spectral energy distribution ($SED$). Note that the average of 
$\delta_m$ over the cataloged area is 0 by construction. 

This error decomposition decouples ``internal absolute'' calibration 
(i.e. producing an internally consistent system by minimizing $\delta_m$), 
from that of ``external absolute'' calibrations (i.e. determining the six 
$ugrizy$ $\Delta_m$  values for the LSST survey). Furthermore, the deviation 
of the LSST system from a perfect AB system, $\Delta_m$, can be expressed 
relative to a fiducial band, say $r$, 
\begin{equation}
            \Delta_m = \Delta_r + \Delta_{mr}.
\end{equation}

The motivation for this separation is twofold. First, $\Delta_{mr}$ can
be constrained by considering the colors (spectral energy distributions) 
of objects, {\it independently from the overall flux scale} (this can 
be done using both external observations and models). Second, there are
few science programs that crucially depend on knowing the ``gray 
scale'' offset, $\Delta_r$, at the 1-2\% level. On the other hand, 
knowing the ``band-to-band'' offsets, $\Delta_{mr}$, with such an 
accuracy is {\it critically important} for many applications (e.g., 
photometric redshifts of galaxies, type Ia supernovae cosmology, testing 
of stellar and galaxy SED models). 






\section{From Flux to Counts}
\label{sec:photons2counts}

We first consider how the photons from an astronomical object make their
way to the detector and are converted into counts (ADUs), paying attention to the various
temporal or spatial scales for variability might arise in the LSST
system to affect the final ADU counts. 

Given $F_\nu(\lambda, t)$ --
the specific flux\footnote{Hereafter, the units for specific
flux (flux per unit frequency  are Jansky (1 Jy = 10$^{-23}$ erg cm$^{-2}$ s$^{-1}$
Hz$^{-1}$). The choice of $F_\nu$ vs. $F_\lambda$ makes the flux
conversion to the AB magnitude scale more transparent, and the choice
of $\lambda$ as the running variable is more convenient than the
choice of $\nu$. Note also, while $F_\nu(\lambda,t)$ (and other
quantities that are functions of time) could vary more quickly than
the standard LSST exposure time of 15s, it is assumed that all such
quantities are averaged over that short exposure time, so that $t$
refers to quantities that can vary from exposure to exposure. }
(flux per unit frequency) of an astronomical object at
the top of the atmosphere -- at a position described by ($alt$,$az$),
the total specific flux from the object transmitted through the atmosphere to the telescope pupil is
\begin{equation}
\label{eqn:Fpupil}
   F_\nu^{pupil}(\lambda,alt,az,t) = F_\nu(\lambda, t) \, S^{atm}(\lambda,alt,az,t),
\end{equation}
where $S^{atm}(\lambda,alt,az)$ is the (dimensionless) probability that a photon of 
wavelength $\lambda$ makes it through the atmosphere,
\begin{equation}
\label{eqn:atmTau}
   S^{atm}(\lambda,alt,az,t)   = {\rm e}^{-\tau^{atm}(\lambda,alt,az,t)}.
\end{equation}
Here $\tau^{atm}(\lambda,alt,az)$ is the optical depth of the
atmospheric layer at wavelength $\lambda$ towards the position
($alt$,$az$). Observational data \citep{Stubbs2007b, Burke2010b} show
that the various atmospheric components which contribute to absorption
(water vapor, aerosol scattering, Rayleigh scattering and molecular
absorption) can lead to variations in $S^{atm}(\lambda,t)$ on the
order of 10\% per hour. Clouds represent an additional gray (non-wavelength
dependent) contribution to $\tau^{atm}$ that can vary even more
rapidly, on the order of 2--10\% of the total extinction at $1^{\circ}$
scales within minutes \citep{Ivezic2007}.

Given the above $F_\nu^{pupil}(\lambda,alt,az,t)$, the total ADU
counts transmitted from the object to a footprint within the field of
view at ($x$, $y$) can be written as
\begin{equation}
\label{eqn:Fpupil2counts}
    C_b(alt, az, x,y,t) = C \, \int_0^\infty {F_\nu^{pupil}(\lambda,alt,az,t) \, S_b^{sys}(\lambda,x,y,t) \lambda^{-1}d\lambda}.
\end{equation}
Here, $S_b^{sys}(\lambda,x,y,t)$ is the (dimensionless) probability
that a photon will pass through the telescope's optical path to be
converted into an ADU count, and 
includes the mirror reflectivities, lens transmissions, filter
transmissions, and detector sensitivities. The term
$\lambda^{-1}$ comes from the conversion of energy per unit frequency
into the number of photons per unit wavelength and $b$ refers to a particular filter, $ugrizy$. The
dimensional conversion constant $C$ is
\begin{equation}
\label{eqn:Cconstant}
        C = {\pi D^2 \Delta t \over 4 g h }  
\end{equation}
where $D$ is the effective primary mirror diameter, $\Delta t$ is the
exposure time, $g$ is the gain of the readout electronics (number of
photoelectrons per ADU count, a number greater than one), and $h$ is
the Planck constant. The wavelength-dependent variations in
$S_b^{sys}$ generally change quite slowly in time; over periods of
months, the mirror reflectance and filter transmission will degrade as
their coatings age. A more rapidly time-varying wavelength-dependent
change in detector sensitivity (particularly at very red wavelengths
in the $y$ band) results from temperature changes in the detector, but
only on scales equivalent to a CCD or larger.  There will also be
wavelength-dependent spatial variations in $S_b^{sys}$ due to
irregularities in the filter material; these are required by the
camera specifications to vary 
slowly from the center of the field of view to the outer edges.  The
equivalent bandpass shift can be no more than 2.5\% of the effective
wavelength of the filter. Wavelength-independent (gray-scale)
variations in $S_b^{sys}$ can occur more rapidly, on timescales of a
day for variations caused by dust particles on the filter or dewar
window, and on spatial scales ranging from the amplifier level,
arising from gain changes between amplifiers, down to the pixel level,
in the case of pixel-to-pixel detector sensitivity variations.

From equation~\ref{eqn:Fpupil2counts} and the paragraphs above, we can
see that the generation of counts $C_b(alt,az,x,y,t)$ from photons is
imprinted with many different effects, each with different variability
scales over time, space, and wavelength. In particular the
wavelength-dependent variability (bandpass shape) is
typically much slower in time and space than the gray-scale variations
(bandpass normalization). These different scales of variability
motivate us to separate the measurement of the normalization of
$S_b^{sys}$ and $S^{atm}$ from the measurement of the
wavelength-dependent shape of the bandpass.

\subsection{Bandpasses and Associated Magnitudes}
\label{sec:phi}

This then leads us to introduce a `normalized bandpass response
function', $\phi_b^{obs}(\lambda,t)$, that represents the true
bandpass response shape for each observation,
\begin{equation}
\label{eqn:PhiDef}
   \phi_b^{obs}(\lambda,t) =  {
     {S^{atm}(\lambda,alt,az,t)\, S_b^{sys}(\lambda,x,y,t) \,
       \lambda^{-1}} \over
     \int_0^\infty { {S^{atm}(\lambda,alt,az,t) \,
         S_b^{sys}(\lambda,x,y,t) \, \lambda^{-1}} \,d\lambda}}.
\end{equation}
Note that $\phi_b$ only represents {\it shape} information about the
bandpass, as by definition
\begin{equation}
\int_0^\infty {\phi_b(\lambda)  d\lambda}=1. 
\end{equation}
Using $\phi_b^{obs}(\lambda, t)$ we can represent the
in-band flux at the top of the atmosphere for each observation as
\begin{equation}
\label{eqn:Fb}
F_b^{obs}(t) = \int_0^\infty {F_\nu(\lambda,t) \,\phi_b^{obs}(\lambda,t) \, d\lambda},
\end{equation}
where the normalization of $F_b(t)$ corresponds to the top of the
atmosphere. Unless $F_\nu(\lambda,t)$ is a flat ($F_\nu(\lambda)=$
constant) SED, $F_b^{obs}$ will vary with changes in
$\phi_b^{obs}(\lambda,t)$ due simply to changes in the bandpass shape,
such as changes with position in the focal plane or differing
atmospheric absorption characteristics, {\it even if the source is
non-variable}.

To provide a reported $F_b^{std}(t)$ which is constant for
non-variable sources, we also introduce the `standardized bandpass response
function', $\phi_b^{std}(\lambda)$, a curve that will be defined before
the start of LSST operations (most likely during
commissioning). $\phi_b^{std}(\lambda)$ represents a typical hardware
and atmospheric transmission curve, roughly minimizing the average difference between
the varying $\phi_b^{obs}(\lambda,t)$ and the standard bandpass.
Now, 
\begin{equation}
\label{eqn:stdFlux}
F_b^{std}(t) = \int_0^{\infty} {F_\nu(\lambda,t) \,
  \phi_b^{std}(\lambda) \, d\lambda}, 
\end{equation}
is a constant value for non-variable sources. 

We define a `natural magnitude' 
\begin{equation}
\label{eqn:natmag}
m_b^{nat}  = -2.5\, log_{10} \left( {F_b^{obs} \over F_{AB}}  \right)
\end{equation}
where $F_{AB}$ = 3631 Jy. The natural magnitude will
vary from observation to observation as $\phi_b^{obs}(\lambda,t)$
changes, even if the source itself is non-variable. The natural
magnitude can be transformed to a `standard magnitude', $m_b^{std}$, as
follows:
\begin{eqnarray}
m_b^{nat} & = &-2.5\, log_{10} \left( {F_b^{obs} \over F_{AB}} \right) \\
& = & -2.5 \, log_{10} \left( { \int_0^\infty {F_\nu(\lambda,t) \,
    \phi_b^{obs}(\lambda,t) \, d\lambda} \over F_{AB} }  \right) \\
& = & -2.5 \, log_{10} \, \left( \left( { \int_0^\infty {F_\nu(\lambda,t) \,
    \phi_b^{obs}(\lambda,t) \, d\lambda} \over \int_0^\infty {F_\nu(\lambda,t) \,
    \phi_b^{std}(\lambda,t) \, d\lambda}} \right) \, \left( {\int_0^\infty {F_\nu(\lambda,t) \,
    \phi_b^{std}(\lambda,t) \, d\lambda} \over F_{AB}} \right) \right) \\
m_b^{nat} & = & \Delta m_b^{obs} + m_b^{std} \\
\label{eqn:Delta_m}
\Delta m_b^{obs} & = & -2.5 \, log_{10} \,  \left( { \int_0^\infty {F_\nu(\lambda,t) \,
    \phi_b^{obs}(\lambda,t) \, d\lambda} \over \int_0^\infty {F_\nu(\lambda,t) \,
    \phi_b^{std}(\lambda,t) \, d\lambda}} \right)
\end{eqnarray}
where $\Delta m_b^{obs}$ varies with the {\it shape} of the source
spectrum, $F_\nu(\lambda,t)$ and the {\it shape} of the bandpass
$\phi_b^{obs}(\lambda,t)$ in each observation. Note that $\Delta
m_b^{obs}=0$ for flat (constant) SEDs, as the integral of
$\phi_b(\lambda)$ is always one.  

The natural and standard magnitudes can be tied back to the counts
produced by the system by adding the correct zeropoint offsets. As
$\Delta m_b^{obs}$ removes all wavelength dependent variations in $m_b^{std}$,
\begin{eqnarray}
\label{eqn:mag2counts}
m_b^{std} & = & m_b^{inst} \, - \Delta m_b^{obs} + Z_b^{obs} \\
 & \equiv & m_b^{corr} + Z_b^{obs} \\
 m_b^{inst} & = & -2.5\,log_{10}(C_b^{obs})
\end{eqnarray}


The zeropoint correction here, $Z_b^{obs}$, contains only gray-scale
{\it normalization} effects, such as variations due to the flat field
or cloud extinction.  The SED corrected magnitude, $m_b^{corr}$, is the input to the Self Calibration block in 
Figure \ref{fig:overview_flowchart1}, and $m_b^{std}$ is its output.

To summarize the various magnitudes utilized, and their associated fluxes:

\begin{itemize}
\item{$m_b^{inst}$, the instrumental magnitude.  $m_b^{inst} = -2.5 log10(C_b^{obs})$, where $C_b^{obs}$ are the instrumental counts (ADU) that are attributed to the object}
\item{$m_b^{nat}$, the natural magnitude.  This is the magnitude in the AB system that would be measured for the object if it were measured through the actual normalized system bandpass, $\phi_b^{obs}(\lambda)$, at the top of the atmosphere.  This bandpass varies
from exposure to exposure.  See equations \ref{eqn:Fb} and \ref{eqn:natmag}. $m_b^{nat} = m_b^{inst} + Z_b^{obs}$}
\item{$m_b^{std}$}, the standard magnitude.  This is the magnitude in the AB system that would be measured for the object if it were measured through the standard normalized system bandpass, $\phi_b^{std}(\lambda)$, at the top of the atmosphere.  This bandpass is 
selected as part of the survey design, and does not vary.  See equation \ref{eqn:stdFlux}. 
\item{$m_b^{corr}$, the SED corrected instrumental magnitude.  This is the standard magnitude, but with an unknown gray zeropoint correction, which will be removed by self calibration.  These magnitudes are the input to self calibration. $m_b^{corr} = m_b^{inst} + \Delta m_b^{obs}$}
\end{itemize}





\vskip 0.3in
\appendix
\leftline{\bf A.1  From Photons to Counts} 

Given $F_\nu(\lambda)$, the specific flux of an object {\it at the top} of the atmosphere,
at a position described by ($alt$,$az$), the flux transmitted through the atmosphere 
to the telescope pupil is 
\begin{equation}
\label{eqn:Fpupil}
                      F_\nu^{pupil}(\lambda,alt,az) = F_\nu(\lambda) \, S^{atm}(\lambda,alt,az),
\end{equation}
where $S^{atm}(\lambda,alt,az)$ is the (dimensionless) probability that a photon of 
wavelength $\lambda$ makes it through the atmosphere,
\begin{equation}
\label{eqn:atmTau}
                        S^{atm}(\lambda,alt,az)   = {\rm e}^{-\tau^{atm}(\lambda,alt,az)}.
\end{equation}
Here $\tau^{atm}(\lambda,alt,az)$ is the optical depth of the atmospheric layer at wavelength 
$\lambda$ towards the position ($alt$,$az$). Note that clouds represent an additive 
contribution to $\tau^{atm}$, and can introduce a strong variation of $\tau^{atm}$ on small 
angular scales (for ``photometric'' atmospheric conditions, most of $\tau^{atm}$ variation is 
a smooth function of $alt$, or airmass). While both $F_\nu(\lambda)$ and $\tau^{atm}(\lambda,alt,az)$ 
can be functions of time, it is assumed hereafter that all quantities are averaged over some 
short (exposure) time. 

Given $F_\nu^{pupil}(\lambda,alt,az)$,  the counts recorded at a position within the field of view
described by ($x$,$y$) can be written as 
\begin{equation}
\label{eqn:Fpupil2counts}
    C_b(alt,az,x,y) = C \, \int_0^\infty {F_\nu^{pupil}(\lambda,alt,az) \, S_b^{sys}(\lambda,x,y) \lambda^{-1}d\lambda}.
\end{equation}
Here, $S_b^{sys}(\lambda,x,y)$ is the (dimensionless) probability that a photon will be converted 
into an ADU count, and the term $\lambda^{-1}$ comes from the conversion of energy per unit frequency 
into the number of photons per unit wavelength ($b=ugrizy$). The dimensional conversion constant $C$ is  
\begin{equation}
\label{eqn:Cconstant}
        C = {\pi D^2 \Delta t \over 4 g h }  
\end{equation}
where $D$ is the effective primary mirror diameter, $\Delta t$ is the exposure time, $g$ is the gain of the 
readout electronics (number of photoelectrons per ADU count, a number greater than one that should
really be called inverse gain), and $h$ is the Planck constant. The system response function, 
$S_b^{sys}(\lambda,x,y)$, includes the (multiplicative) effects of the mirror reflectance, transmission of
the lenses and filters, and efficiency of the camera sensors. It is assumed hereafter that the
pixel-to-pixel variation of the sensor efficiency is treated via flatfields, and that the dependence 
of $S_b^{sys}(\lambda,x,y)$ on ($x$,$y$) refers to scales sufficiently larger than a seeing disk. 

{\it Note that this assumption implies that flatfield variation on small spatial scales
is not a function of wavelength!}  In more quantitative detail, it is assumed that
the instrument response $T$ is a separable function of position and wavelength:
\begin{equation}
               T_b^{sys}(x,y,\lambda) = FF_b(x,y) \, S_b^{sys}(\lambda), 
\end{equation}
where $(x,y)$ are pixel coordinates. Here we can define $FF_b(x,y)$ to be
self-normalized, and its main purpose is to flatten the pixel-to-pixel
response on small (up to CCD size) scales.

Of course, $S_b^{sys}(\lambda)$  is also allowed to vary with position,
but only on angular scales much {\it larger} than the seeing disk. That is, we are
separating the small-scale (sub-seeing disk) variation into $FF_b$ and the
large-scale variation that {\it we can control with stars} into $S_b^{sys}$. 

The assumption that the small-scale $FF_b$ variation is not a function of wavelength
enables the extraction of counts by DM pipelines without knowing the color of a source. 
Even if this is not strictly true, we should be able to correct for these small
effects a posteriori (and stars will be observed at different positions
so the impact on an ensamble of measured should be minimal).

The extracted counts depend on the applied aperture weighting scheme. In the context of 
photometric calibration discussed here, only point sources with optimally extracted counts
are considered. In particular, it is assumed that aperture correction, which corrects the
so-called point-spread-function counts to an infinitely large aperture will be determined
as part of regular image processing by DM pipelines. Of course, the impact of errors in aperture
correction on resulting photometric calibration will also be studied with this framework
(beginning with the analysis of DC3b outputs). 



\vskip 0.3in
\leftline{\bf A.2 The Basics of Photometric Calibration}

The previous section details how counts are produced in the camera; this section discusses how we
can calibrate those counts (which vary from observation to observation and source to source) to 
standardized, calibrated flux measurements for each source in each observation. This is best
served by introducing a new quantity $\phi_b(\lambda)$, the normalized system response function in bandpass $b$,
\begin{equation}
\label{PhiDef}
   \phi_b(\lambda) \equiv {\lambda^{-1} S^{atm}(\lambda) S_b^{sys}(\lambda) \over \int_0^\infty {\lambda^{-1} S^{atm}(\lambda) S_b^{sys}(\lambda) d\lambda}}.
\end{equation}
By definition, $\int_0^\infty {\phi_b(\lambda) d\lambda}=1$. It is implied that
$\phi_b(\lambda)$ is a function of position.  

In addition to providing a compact description for in-band flux (see Eq.~\ref{eqn:Finband} 
below), $\phi_b(\lambda)$ conveniently separates the
shape of the response function from its normalization, a separation
that maps well onto the baseline calibration strategy.  The shape of 
the response functions will be determined by direct measurements
using auxiliary instrumentation, while the normalization (also known
as `gray' zeropoint calibration) will be determined self-consistently
using the main survey data for non-variable main-sequence stars. 

For given $S^{atm}(\lambda)$ and $S_b^{sys}(\lambda)$, the in-band flux can be 
defined as 
\begin{equation}
\label{eqn:Finband}
                          F_b = \int_0^\infty {F_\nu(\lambda) \phi_b(\lambda) d\lambda}.
\end{equation}
The normalization of $F_b$ corresponds to the top of the atmosphere,
but of course the atmosphere affects the wavelength dependence (or shape) of 
$\phi_b(\lambda)$. Note that gray effects do not change $F_b$, due to the 
definition of $\phi_b(\lambda)$ as the normalized system response.  For
convenience, we define corresponding magnitude as 
\begin{equation}
\label{eqn:maginband}
                    m_b = -2.5\log_{10}(F_b / F_{AB}),
\end{equation}
with the flux normalization $F_{AB} = 3631$ Jy (1 Jansky = 10$^{-26}$ W Hz$^{-1}$ m$^{-2}$ = 
10$^{-23}$ erg cm$^{-2}$s$^{-1}$ Hz$^{-1}$). 


Even for perfectly calibrated data, and for perfectly non-variable 
stars, $F_b$ will vary between different observations since in general 
$\phi_b(\lambda)$ will vary due to changing atmospheric conditions  
(and perhaps due to hardware aging or changes in the instrumental setup). 
To correct $F_b$ for changes in $\phi_b(\lambda)$ (e.g. to derive
standardized fluxes or to measure low-amplitude variability), both 
the shape of the source
SED, $F_\nu(\lambda)$, and $\phi_b(\lambda)$ must be known. 
For example, a source with a flat SED, $F_\nu(\lambda)=F_o$, would 
have $F_b = F_o$ regardless of the wavelength dependence of $\phi_b(\lambda)$, 
but a power-law SED with more flux in the blue than the red results in a larger
$F_b$ when $\phi_b(\lambda)$ also has a higher response in the blue.

The aim of photometric calibration is to calibrate the counts $C_b(alt,az,x,y)$ measured in each observation 
to the corresponding $F_b$, and to provide a measurement of $\phi_b(\lambda)$ for that particular observation. 
Such pairs of ($F_b$,$\phi_b(\lambda)$) will represent the main photometric products --- the 
survey will collect $\sim$ 10$^{13}$ such pairs over a ten year period: one pair for each observation 
($\sim$1000) of each source ($\sim10^{10}$). 







\end{document} 


