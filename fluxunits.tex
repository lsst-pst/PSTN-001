\documentclass[DM,lsstdraft,toc,usenatbib]{lsstdoc}

% Package imports go here
\usepackage{amsmath}	% Advanced maths commands
\usepackage{amssymb}
\usepackage{gensymb}  % degree symbol
\usepackage{natbib}  % bibliography
\usepackage{cprotect}
% Local commands go here

%% Journal abbreviations
%\bibliographystyle{aasjournal}

\title[LSST Flux Units]{On the Choice of LSST Flux Units}

\author{\v{Z}eljko Ivezi\'{c} and the LSST Project Science Team}

\setDocRef{Document-27758}
\date{2018-02-14}
\setDocAbstract{%
A linear measure of flux is preferred for LSST catalogs. This document
provides technical details about this preference in support of the LSST Project
Science Team's decision to adopt nano-jansky (1 nJy $=10^{-35} \,\, \mathrm{W m^{-2} Hz^{-1}}$)
as the standard LSST flux unit. Difficulties associated with homogenizing broad-band
flux measurements to a uniform system are also discussed in some detail.
}

% Change history defined here. Will be inserted into
% correct place with \maketitle
% OLDEST FIRST: VERSION, DATE, DESCRIPTION, OWNER NAME
\setDocChangeRecord{%
\addtohist{1.0}{2018-02-14}{First released version.}{\v{Z}eljko Ivezi\'{c}}
}

\begin{document}

% Create the title page
% Table of contents will be added automatically if "toc" class option
% is used.
\maketitle


\section{Introduction}

A linear measure of flux, not logarithmic magnitude scale, is preferred
for LSST catalogs (e.g., forced fluxes can be negative due to stochastic
background fluctuations). LSST flux measurements will be obtained using
time-dependent and focal plane position-dependent broad photometric
bandpasses. As a crucial undesired consequence of these bandpass variations
and the broadness of the bandpasses, even intrinsically constant non-variable
sources may display flux variability at the level exceeding LSST's photometric precision
and accuracy requirements.

At the precision level of these requirements ($\sim1\%$), it is indeed
\textit{mathematically impossible} to define a natural LSST photometric system (or
natural magnitudes): in order to homogenize broad-band flux measurements
to a uniform system, \textit{assumptions about the shapes of the source spectral
energy distribution must be made!}

These fundamental issues are often conflated with discussions of differences
between AB and Vega photometric systems, and choices of magnitudes vs. linear
fluxes, sometimes yielding confusing statements.  This document aims to clarify
some of the ensuing confusion by summarizing most important technical details
regarding interpretation of broad-band photometry and systematic uncertainties in
photometric calibration. It also provides supporting material for the LSST Project
Science Team's decision to adopt nano-jansky as the standard LSST flux unit.

Relevant technical discussion is provided in \S 2. Readers familiar
with broad-band photometry and photometric calibration can skip
directly to \S 3, where justification for adopting nano-jansky
as the LSST flux unit is discussed and summarized.



\section{What flux will LSST measure?}

The complexity of the interpretation of measured and calibrated flux depends on
the required flux precision and accuracy, as well as the width of the photometric bandpass.
Were the photometric bandpasses infinitely narrow, or if the precision and accuracy
requirements could be relaxed by a factor of a few, much of discussion presented in this
document would be moot.

In this Section we review definitions of the basic quantities and briefly discuss some of the
consequences of the broad photometric bandpasses. A more detailed discussion is
available in the LSST document \citeds{LSE-180} (Level 2 Photometric Calibration for the LSST Survey).


\subsection{CCDs count photons}

CCDs don't measure energy flux -- CCDs count photons over some wavelength range set
by the overall atmosphere plus system throughput,  $S_b(\lambda)$ (defined as the probability
that a photon with wavelength $\lambda$, or frequency $\nu=c/\lambda$,  will be
transmitted through Earth's atmosphere and through the observing apparatus, and converted
into an electron, including the effects of both optics, sensors and all other potential losses).
This quantity is \textit{not}  defined per unit energy, or per unit wavelength (or frequency):
it is simply a dimensionless probability -- a number between 0 and 1.

Given a flux of photons per unit time, area and frequency interval, $N_\nu$, the
source counts, $C_b$ (in ADU), are proportional to
\begin{equation}
\label{Nnu}
        C_b \propto \int N_\nu(\nu) \, S_b(\nu) \, d\nu,
\end{equation}
with the constant of proportionality increasing with effective collecting area and exposure
time (index $b$ stands for ``bandpass''). The fact that CCDs count photons is reflected in
the integration of the photon flux, $N_\nu$; in case of a calorimeter, for example, energy
flux (i.e., the specific flux, see the next subsection) would be integrated.

The integration in eq.~\ref{Nnu} is over frequency $\nu$ because the photon flux $N_\nu$
is expressed per unit frequency. However, the running variable can be either  $\nu$ or
$\lambda$, and the choice of $\lambda$ is more convenient in this context. It is
easy to show, using $|d\nu/\nu| = |d\lambda / \lambda|$ (which follows from $\nu \lambda=c$),
that an equivalent form of  eq.~\ref{Nnu} is
\begin{equation}
\label{Nnu2}
         C_b\propto \int N_\nu(\lambda) \, S_b(\lambda) \, \lambda^{-2} \, d\lambda.
\end{equation}


CCDs count photons but (unfortunately) don't record the photons' wavelength/frequency/energy.
Nevertheless, it is possible to relate (calibrate) the measured source counts to energy flux -- though
with a few important caveats, as described below. It is important to emphasize, however, that the main
source of complexity for LSST photometry is the combination of its required exquisite precision
and substantial photometric bandpass width, rather than the fact that CCDs count photons. If LSST
used a calorimeter device instead, the problems of measurement homogenization to account for varying
bandpass would persist.



\subsection{Definition of the specific flux}

Let us first  define $F_\nu$: the specific flux (flux per unit frequency, $\nu$) of an
object \textit{at the top} of Earth's atmosphere. The SI units for $F_\nu$ are
W m$^{-2}$ Hz$^{-1}$ (= 10$^{3}$ erg cm$^{-2}$ s$^{-1}$). Because astronomical
fluxes are small, the IAU defined in 1973 a more convenient unit\footnote{The
name honors radio astronomer Karl Jansky. Papers about the 3$^{rd}$ Cambridge
catalog of quasars published in 1950's already used $10^{-26} \, \mathrm{W m^{-2} Hz^{-1}}$
as a standard flux unit (without calling it jansky).},  jansky (Jy):
\begin{equation}
            1 \, \mathrm{Jy} = 10^{-26} \,\, \mathrm{W m^{-2} Hz^{-1}}.
\end{equation}

The specific flux can also be defined per unit wavelength, $F_\lambda$, using energy
conservation $F_\nu |d\nu|=F_\lambda |d\lambda|$ and $\lambda \nu = c$.
The choice of $F_\nu$, as opposed to $F_\lambda$, makes the flux conversion
to the AB magnitude scale (see below) more transparent, but otherwise is completely
arbitrary. Similarly, the running variable can be either  $\lambda$ or $\nu$, and the
choice of $\lambda$ is more convenient.



\subsection{From counts to the specific flux}

The flux of photons, $N_\nu$ (s$^{-1}$ m$^{-2}$ Hz$^{-1}$), is related to the specific flux $F_\nu$ as
\begin{equation}
\label{FNnu}
                 F_\nu = h \nu \times N_\nu,
\end{equation}
where $h\nu$ is the energy of a single photon with frequency $\nu$. Therefore,
eq.~\ref{Nnu2} can be rewritten as
\begin{equation}
\label{Nnu3}
         C_b\propto \int F_\nu(\lambda) \, S_b(\lambda) \, \lambda^{-1} \, d\lambda.
\end{equation}

If instead of CCDs which measure $N_\nu$, LSST used a calorimeter device that measures $F_\nu$,
eq.~\ref{Nnu3} would have a $\lambda^{-2}$ term instead of $\lambda^{-1}$.  Again, most calibration
difficulties are not due to the fact that CCDs count photons but come from the fact that $S_b(\lambda)$
is a broad function, it is variable, and the shape of source spectral energy distribution is not known
a priori.

\subsection{Astronomical magnitudes \label{sec:tradmag}}

Counts $C_b$ have to be calibrated to be useful for science. Ideally, the reported flux measurements
of a non-variable source should be constant, up to random noise. For illustration, the three most common
use cases are
\begin{enumerate}
\item How can we produce the crispest possible color-magnitude and color-color diagrams?
\item How can we best recognize a low-amplitude variable source?
\item Given a model flux $F_\nu(\lambda)$ (possibly as a function of time), how can it be best
          compared to measurements?
\end{enumerate}

Traditional astronomical magnitudes are reported as a measurement relative to some judiciously
chosen celestial calibration source (e.g., Vega),
\begin{equation}
\label{Vegamag}
               m^{Vega}_b = -2.5\log_{10}\left({C_b \over C^{calib}_{b}}\right),
\end{equation}
where $C_b^{calib}$ is the counts measurement for a calibration star\footnote{In practice,
as opposed to this formal definition, multiple stars are used to derive the calibration
zeropoint, and the measurement noise is taken into account as well.} obtained with the
same system (that is, with the same $S_b(\lambda)$ up to a normalization constant) as $C_b$.

The throughput $S_b(\lambda)$ cannot be assumed constant because of varying observing
conditions (even if the system and Earth's atmosphere are assumed constant in time,
$S_b(\lambda)$ will depend on airmass). As $S_b(\lambda)$ varies, $m^{Vega}_b$
would vary too (beyond random measurement noise) even for a non-variable source (unless
it has exactly the same spectral energy distribution as the calibration source)!

This clearly undesireable effect is in practice corrected for using the so-called ``color terms''.
Therefore, eq.~\ref{Vegamag} is deceivingly simple and the need to \textit{assume} the
shape of the source's spectral energy distribution, as discussed below, \textit{cannot be avoided}.
Further details can be found in \citeds{LSE-180} (see Section 4.1).


\subsection{Definition of broad-band flux}

As discussed in the LSST Science Requirements Document (Section 3.3.4), ``LSST is a
broad-band photometric system and will deliver calibrated in-band flux, $F_b$,
defined by
\begin{equation}
\label{Fb}
              F_b = \int{F_\nu(\lambda) \phi_b(\lambda) d\lambda},
\end{equation}
where $F_\nu(\lambda)$ is the specific flux of an object \textit{at the top} of the atmosphere,
and $\phi_b(\lambda)$ is the normalized system response for the given band.'' This
expression follows directly from eq.~\ref{Nnu3}. The calibration of measured $C_b$ to obtain
$F_b$ is addressed further below.

The normalized system response is defined by
\begin{equation}
\label{PhiDef}
\phi_b(\lambda) = {S_b(\lambda) \lambda^{-1} \over \int{S_b(\lambda) \lambda^{-1} d\lambda}}.
\end{equation}
where $S_b(\lambda)$ is the overall atmosphere + system throughput.  It is defined by
\begin{equation}
\label{SDef}
         S_b(\lambda) = S_b^{sys}(\lambda) \times S_b^{atm}(\lambda),
\end{equation}
where $S_b^{atm}(\lambda)$ is the \textit{probability} that a photon
with wavelength $\lambda$ will be transmitted through the atmosphere, and
$S_b^{sys}(\lambda)$ is the probability that the transmitted photon will be
converted into an electron by the system (optics, CCDs). Again, these quantities
are dimensionless probabilities (numbers between 0 and 1).

As discussed above, the $\lambda^{-1}$ factor in eq.~\ref{PhiDef} reflects the fact that CCDs are
photon-counting devices: it comes from the conversion of energy per unit frequency
into the number of photons per unit wavelength. If a calorimeter was used instead, $\lambda^{-1}$
term would turn into $\lambda^{-2}$ but all the problems caused by integration over the
broad bandpass would remain.

Note that the product $\phi_b(\lambda) d\lambda$ is dimensionless; it acts in eq.~\ref{Fb} as a
dimensionless weighting function and the unit for $F_b$ is \textit{same} as for $F_\nu$
(by construction).  In other words, eq.~\ref{Fb} doesn't represent ``integration under the
$F_\nu$ curve'' -- instead, $F_b$ is a ``weighted average'' of $F_\nu$. While details of the
``weighting function''  $\phi_b(\lambda)$ depend on the device properties, the fact
that $F_\nu$ is a ``weighted'' quantity is purely a consequence of the broad bandpass.
Note that only in the case of an infinitely narrow bandpass, the measured quantity would
really be the specific flux $F_\nu$ at some well defined wavelength.

Had we chosen to use $F_\lambda$ instead of $F_\nu$, an analog
to eq.~\ref{Fb} would have been
\begin{equation}
\label{FbL}
              F_b^\ast = \int{F_\lambda(\lambda) \phi^\ast_b(\lambda) d\lambda},
\end{equation}
where
\begin{equation}
\label{PhiL}
\phi^\ast_b(\lambda) = {S_b(\lambda) \lambda \over \int{S_b(\lambda) \lambda d\lambda}}.
\end{equation}
In this case, the calibrated flux $F_b^\ast$ has the same units as $F_\lambda$, and
the weighting factor $\phi^\ast_b(\lambda) d\lambda$ is now skewed more towards
the red edge of the bandpass, compared to $\phi_b(\lambda) d\lambda$. Of course,
$C_b$ is same in both cases -- it is only our (arbitrary) choice of flux calibration that
distinguishes $F_b$  and $F_b^\ast$. The only practical implication of the choice between
$F_b$  and $F_b^\ast$ is the type of the spectral energy distribution for which photometric
standardization correction for bandpass variation (see eq.~\ref{FbStd}) vanishes.


\subsection{The curse of broad-band flux}

There are a few consequences of the finite width of $\phi_b(\lambda)$ that need to be
(re)emphasized (and are \textit{not} the result of the specific choice for flux calibration).

Even for a temporally non-variable $F_\nu(\lambda)$,  $F_b$ will vary if
$\phi_b(\lambda)$ varies (even if the atmospheric and system properties
are unchanged, variation of observing airmass can change $\phi_b(\lambda)$).
In order to standardize measurements to a
common standard system, one in which $F_b$ of a temporally non-variable
source would not vary (modulo random noise),  we need to define a \textit{standard}
normalized system response, $\phi^{std}_b(\lambda)$. In addition, we must know the \textit{shape}
of the source spectral energy distribution (SED), $f_\nu(\lambda)$, defined by
\begin{equation}
        F_\nu(\lambda) = F_o \, f_\nu(\lambda),
\end{equation}
where $F_\nu(\lambda_o) = F_o$ and $f_\nu(\lambda_o) = 1$ for some fiducial
wavelength $\lambda_o$. Then we can compute standardized flux as
\begin{equation}
\label{FbStd}
F^{std}_b =  F_b \, {\int{f_\nu(\lambda)  \phi^{std}_b(\lambda) d\lambda }  \over   \int{ f_\nu(\lambda) \phi_b(\lambda) d\lambda }}.
\end{equation}
Traditionally, corrections to the standard system (the ratio of two integrals in eq.~\ref{FbStd})
are called \textit{color terms}. Historically, they were obtained empirically (typically as approximating
linear functions of source color and airmass) rather than by using eq.~\ref{FbStd}. Assuming
main-sequence stars and standard atmosphere, plausible variations of airmass induce variations
of $F_b$ around $F^{std}_b$ of a few percent (i.e., several times larger than the photometric
precision requirements).

Without knowing, or assuming, $f_\nu(\lambda)$, it is \textit{mathematically impossible}  to
standardize measurements $F_b$ -- this is the ``curse'' of broad-band fluxes
(note that in the special case of a flat SED, $f_\nu(\lambda)$ = constant, $F^{std}_b =  F_b$;
had we chosen $F_b^\ast$ instead of $F_b$, the standardization correction would vanish for
$f_\nu(\lambda) = \lambda^2$, that is, for $f_\lambda(\lambda)$ = constant).
This ``curse'' cannot be avoided whatever is the flux calibration choice (i.e. $F_\nu$ vs. $F_\lambda$);
the ``problem'' is the finite width of the photometric bandpass.


\subsection{Standardized fluxes}

For each flux measurement, LSST will report both $F_b$ and $\phi_b$ (with $\phi^{std}_b(\lambda)$
pre-defined and always known). There is also need to report standardized flux
computed using eq.~\ref{FbStd} (e.g., to help users construct color-color and color-magnitude
diagrams, or to search for variable sources). There are at least two options for choosing $f_\nu(\lambda)$:
i) assume a flat SED ($F^{std}_b =  F_b$, i.e. effectively no correction is applied), and ii) assume
the best possible estimate of object's SED, using available LSST color measurements and possibly
other information.

Neither choice is perfect: the first choice, while simple, does not account for the variation of $F_b$ around
$F^{std}_b$ due to changes of $\phi_b$ (except in case of flat SED), while the second choice
can be grossly incorrect (e.g., when SED type is incorrectly chosen, such as stellar SED instead
of quasar or supernova SED). Therefore, it is important to enable users i) to undo whatever default
flux standardization correction was used, and ii) to easily re-do the computation with a different choice
of the spectral energy distribution (e.g., for multiple hypothesis testing, such as distinguishing
``star'', ``quasar'', and ``supernova'' SEDs, or galaxy SEDs of different intrinsic types and at
different redshift). The current baseline LSST plan is option ii).

\subsection{Some pitfalls when interpreting measured fluxes}

When comparing a model for the specific flux, $F_\nu^{model}(\lambda)$, to
measurements $F_b$, the proper way to proceed is to compute the model prediction
for $F_b$ using eq.~\ref{Fb}
\begin{equation}
\label{FbModel}
             F^{model}_b = \int{F_\nu^{model}(\lambda) \phi_b(\lambda) d\lambda},
\end{equation}
and then compare $F^{model}_b$ to measurement $F_b$.
When measurements $F_b$ have already been standardized as $F^{std}_b$,
this data vs. model comparison (e.g., photometric redshift estimation) can
suffer from systematic errors when SED shape, $f_\nu(\lambda)$,  assumed
for flux standardization differs from the shape of $F_\nu^{model}(\lambda)$,
\textit{even when} $\phi_b(\lambda)$ was substituted by $\phi^{std}_b(\lambda)$.

Flux measurements ($F_b$ or $F^{std}_b$) are often interpreted as corresponding to
$F_\nu(\lambda_{eff})$, at some effective wavelength, $\lambda_{eff}$, and compared
to model flux $F_\nu^{model}(\lambda_{eff})$. This practice usually results in systematic
errors because $\lambda_{eff}$ is a function of $f_\nu(\lambda)$, and thus there
is no universal value of $\lambda_{eff}$ applicable to all sources.


\subsection{Calibration of counts to fluxes}

Image processing pipelines, more precisely object measurement pipelines/algorithms,
will produce counts, $C_b$ (together with $\phi_b$). It is assumed that all the relevant
instrumental and other effects had been taken into account such that the following
relationship is valid
\begin{equation}
               F_b = \alpha \,  C_b,
\end{equation}
for all sources from some judiciously chosen ``calibration patch'' (spatial variation of $\alpha$ over
the patch is assumed to be corrected for, though the above equation could be easily
generalized; in principle, this  ``calibration patch'' could correspond to the entire
sky if all $C_b$ were ``reduced to the same system'' using self-calibration).

A set of calibration stars will be available to estimate $\alpha$: for these stars
we (assume that we) will know broad-band flux in an arbitrary calibration bandpass
$\phi_{calib}(\lambda)$ (e.g., in Gaia's bandpasses)
\begin{equation}
\label{FbCalib}
             F_{calib} = \int{F_\nu(\lambda) \phi_{calib}(\lambda) d\lambda}.
\end{equation}
We also assume that we will have a good knowledge of the SED shape,
$f_\nu(\lambda)$ for calibration stars. The implied fluxes of calibration stars corresponding
to bandpass $b$, $F_b^{calib}$ can then be computed analogously to eq.~\ref{FbStd}, using
$\phi_b(\lambda)$ and $\phi_{calib}(\lambda)$,
\begin{equation}
\label{FbStdCalib}
F^{calib}_b =  F_{calib} \, {\int{f_\nu(\lambda)  \phi_b(\lambda) d\lambda }  \over   \int{ f_\nu(\lambda) \phi_{calib}(\lambda) d\lambda }}.
\end{equation}


Finally, the calibration coefficient $\alpha$ (related to photometric zeropoint when
working in magnitude space) is computed from
\begin{equation}
\label{alphaCalib}
                 F_b^{calib} = \alpha \,  C_b^{calib},
\end{equation}
by the usual least squares minimization, or perhaps using a more robust statistical method.


\subsection{AB magnitudes}
The in-band flux in astronomy is often reported on a magnitude scale, and LSST has adopted
AB magnitudes defined as
\begin{equation}
\label{ABmag}
              m^{AB}_b = -2.5\log_{10}\left({F_b \over F_{AB}}\right).
\end{equation}
where $F_{AB}$ = 3631 Jy. The same expression applies to $F^{std}_b$, or any other flux.
The choice of normalization flux $F_{AB}$ results in correspondence between AB magnitudes
and Vega magnitudes in the Johnson's $V$ band (i.e., approximately, 3631 Jy is the flux of Vega
in the standard V band).


\subsection{Relationship between AB magnitudes and Vega magnitudes}

It follows from eqs.~\ref{Vegamag} and \ref{alphaCalib} that
$F_b/F^{calib}_{b} = C_b / C_b^{calib}$ and
\begin{equation}
\label{VegamagAB}
               m^{Vega}_b = -2.5\log_{10}\left({F_b \over F^{calib}_{b}}\right).
\end{equation}
When $F^{calib}_{b}$ is available (e.g., Vega's flux of 3631 Jy in the V band), transformations between
$m^{AB}_b$ and $m^{Vega}_b$ are straightforward and effectively there is no practical difference
between the two approaches. When $F^{calib}_{b}$ is unavailable, $m^{Vega}_b$ are simply relative flux measurements
and cannot be transformed to an absolute flux scale implied by AB magnitudes.


\section{The choice of flux unit}

Most astronomical surveys, especially space-based surveys and surveys
at wavelengths other than optical, used jansky as the flux unit.  One notable
exception is SDSS, which used maggies (see Appendix A). There appears to be
no advantage for LSST to adopt maggies over jansky, and for consistency
with the practice adopted by numerous other astronomical catalogs, it seems
better to adopt jansky.


\subsection{Adopting jansky as the preferred flux unit for LSST  \label{sec:Jy}}

There is fundamentally no difference between the flux measurements discussed
here and any other physical measurement that is subject to random and systematic
uncertainties. It is rather common to report both types of measurement uncertainties
in physical sciences. Indeed, the LSST  Science Requirements Document (Section 3.3.4)
explicitly addresses the issue of systematic uncertainties in photometry and
introduces the separation of ``internal absolute'' calibration accuracy
from ``external absolute'' ” calibration accuracy and from the offset from the
overall flux scale. The deviation of the LSST system from a perfect AB system
(that is, the systematic error in calibrating the flux scale to correspond to jansky)
$\Delta_b$, is expressed relative to the fiducial band (chosen to be the $r$ band)
as \begin{equation}
             \Delta_b = \Delta_r + \Delta_{br},
\end{equation}
(eq.~9 from the SRD), where $\Delta_r$ describes overall systematic uncertainty
in the LSST flux scale calibration (a single number for the whole survey). The
``band-to-band zeropoint errors'' $\Delta_{br}$ are thus decoupled from the overall
``gray scale'' offset $\Delta_{r}$, which minimizes error covariances. Similarly to SDSS,
a variety of methods will be used to assess likely values of $\Delta_{br}$ and
$\Delta_r$ for each LSST Data Release.


\subsection{Summary}

It is advocated here to calibrate LSST fluxes to a physical scale (e.g., using
Gaia catalogs), adopt jansky (Jy) as the LSST flux unit, and report the best available
estimates of $\Delta_{br}$ and $\Delta_r$ with each LSST Data Release. In particular,
$\Delta_r$ will describe the systematic flux uncertainty --  the discrepancy between an
ideal flux scale in jansky and ``jansky'' scale reported by LSST.

For the faint fluxes probed by LSST, a convenient unit is nano-jansky (nJy).
The AB magnitude values of 27.5 (fiducial coadded image depth)  and 24.5 (fiducial
single-image depth) correspond to 36.3 nJy and 575 nJy.

Lastly, we will need to strongly emphasize to the users that LSST measurements are
fundamentally broad-band fluxes, and not ``a flux-at-a-wavelength'', despite using the
unit for the specific flux. In addition, sufficient information will have to be provided
in data products (catalogs and documentation) so that the entire calibration process from
source counts in ADU to fluxes in nano-jansky can be easily understood and back-engineered
if desired.


\vskip 0.0in
\newpage
\textit{Acknowledgments:} this document has greatly benefited from discussions between
the LSST Project Science Team members and Tim Axelrod, Michael Blanton, David
Burke, Arjun Dey, Mark Dickinson, Doug Finkbeiner, Tim Jenness, Lynne Jones,  Jeff Newman,
John Parejko, David Schlegel, and Peter Yoachim.

\bibliography{lsst,refs_ads}

\appendix

\section{The curious case of maggies}

In analogy with eqs.~\ref{Vegamag} and \ref{ABmag}, magnitudes can be defined\footnote{
See \url{http://www.sdss3.org/dr8/algorithms/magnitudes.php}} as
\begin{equation}
\label{maggie}
               m \equiv - 2.5\log_{10}\left(\mathrm{maggie}\right),
\end{equation}
where ``maggie'' is the source flux expressed in some arbitrary
units,
\begin{equation}
\label{maggie2}
               \mathrm{maggie} \equiv {F_b \over F_o}.
\end{equation}
In case of AB magnitudes, eq.~\ref{ABmag} implies that maggie is
flux measured in units of 3631 Jy,
\begin{equation}
\label{eq:defmaggie}
           \mathrm{maggie} = {\mathrm{flux \, (Jy)} \over \mathrm{3631 \, Jy}}.
\end{equation}
In practice (e.g., SDSS), a more convenient
quantity is nano-maggie, which is flux measured in units of 3631 nJy.

In case of traditional magnitudes (see \S\ref{sec:tradmag}), maggies are a
relative flux measure
\begin{equation}
\label{maggie3}
    \mathrm{maggie} \equiv {C_b  \over C_b^{calib}} = {F_b  \over F_b^{calib}}.
\end{equation}

Maggies were introduced as an alternative flux unit in order to emphasize
that zeropoints for astronomical flux calibration change often -- and
presumably much more often than the actual counts measurement. Such
was the case of SDSS, where the sky was imaged essentially once and with
a precision better then the accuracy of calibration photometry (i.e., at
the bright end systematic photometric uncertainties were larger than
random uncertainties). It was anticipated that photometric zeropoints
would eventually improve and the dataset recalibrated.  But it needs to
be noted that fluxes also change when other aspects of calibration change
(non-linearity and cross-talk corrections, flatfields, standard apertures,
point-spread function, etc.) -- photometric zeropoints are only one of
many calibration factors. Indeed, it turned out that in practice most
SDSS users would simply convert maggies to AB magnitudes (by assuming
that $F_o$ from eq.~\ref{maggie2} is 3631 Jy), and thus
implicitly to jansky (and sometimes explicitly, e.g., when constructing
multi-wavelength spectral energy distributions). SDSS Project curates a
list of five best estimates of the systematic flux errors in the $ugriz$
bands introduced by this conversion. These errors are analogous to
quantities $\Delta_b$ discussed in Section~\ref{sec:Jy}.

It is sometimes claimed that maggies have the benefit of not pretending
to exactly correspond to physical units (Jy). This somewhat philosophical
advantage was not born in practice because of the immediate users'
conversion to AB magnitudes mentioned above.


\section{Discussion}

The following few points that often came up in previous discussions are worth reiterating:
\begin{enumerate}
\item There isn't that much difference between using maggies (when expressed on an
absolute flux scale) and jansky: the key decision is to report physical flux on a linear scale.
\item Irrespective of which flux unit is chosen, the changes of the photometric
calibration zeropoints will affect whatever is reported, whether it is maggie, $mag^{AB}$,
$mag^{Vega}$, or nJy, as long as it is implied that physical flux is reported. Only if
truly relative measures of flux are reported (e.g., with respect to Vega, but without knowing
what calibrated Vega fluxes really are), will magnitudes (but not fluxes in maggies or
nJy) stay unchanged. However, in this case rather ugly details are hidden (e.g., how
exactly was the relative flux calibrated and how stable is the flux scale), and in the context
of sub-1\% photometry this case appears to be of historical interest only. After all,
LSST SRD mandates that reported fluxes are calibrated on an absolute flux scale.
\item ``But: we are measuring broad-band flux, and not the specific flux!''  Well,
yes, this is an unfortunate fact. However, it is not relevant for the maggie vs. Jy discussion.
It is impossible (mathematically!) to relate these two flux measures without knowing
the source SED, whether we use relative or absolute flux calibration and magnitudes,
maggies or Jy!
\item Once an estimate of the source SED is available or assumed, the transformation from
the broad-band flux to the specific flux is dimensionless and thus obviously independent
of the choice for flux units and relative vs. absolute calibration choice (again, the broad band
flux vs. the specific flux issues do not provide arguments for maggie vs. Jy discussion).
\end{enumerate}



\end{document}
